% comment out for student version
\ifdefined\Student\relax\else\def\Teacher{}\fi

\documentclass[12pt]{article}

\title{2D Arrays and Recursion}
\author{Ian Ludden}
\date{Summer 2025}

\input{../../cspogil.sty}

\begin{document}

\maketitle

% Today, we start with reading user input from the console, or command line. 
% We then examine encapsulation, one of the pillars of object-oriented design. 

\ifdefined\Student\rolenames{}\fi

\guide{
    \item TODO
    % \item I can read various data types %(string, integer, floating point) 
    %     from the command line using the built-in \java{Scanner} class. 
    % \item I can explain the difference between 
    %     the \java{next()} and \java{nextLine()} Scanner methods. 
    % \item I can add new commands to a command-line interface. 
    % \item I can use GenAI tools to rapidly prototype software designs. 
    % \item I can explain encapsulation and evaluate how well given UML designs achieve it. 
}{
    \item TODO
    % \item Identifying similarities and differences between syntax for 
    %     reading different data types from the command line. (Problem Solving)
    % \item Interpreting UML class diagrams. (Information Processing)
    % \item Understanding the context of a problem/system and 
    %     applying software design principles. (Problem Solving)
}{
TODO: facilitator notes
% \textbf{First hour, before quiz:}
% \ref{scanner.tex} introduces console input using the \java{Scanner} class. 
% Mention that console input will appear in HWVaporSales. 
% \ref{cli.tex} builds on this to talk about command-line interfaces and walks through 
% an example using the familiar Animal Shelter system. 
% In \ref{removeAnimal}, students pair-program a new command, 
% which has the intended side-effect of reminding them how to remove entries from HashMaps. 

% \textbf{Second hour:}
% \ref{prototyping-genai.tex} demonstrates using GenAI tools 
% to quickly prototype different UML designs 
% and see their strengths/weaknesses more clearly. 
% \ref{encapsulation.tex} introduces encapsulation (Design Principle 3) 
% and gives teams opportunities to practice applying it, 
% including one in which they use GenAI tools to prototype alternate designs. 

Key questions: TODO
% \ref{scannerTokens}, 
% \ref{removeAnimal}, 
% \ref{bookMainDiffs}, 
% \ref{functionalButBad}, 
% \ref{encapsAnalogy}, 
% \ref{pizza}

Source files: TODO
% \src{Act05}{ConsoleInputMain.java},  
% \src{Act05}{Animal.java},
% \src{Act05}{AnimalShelterMain.java}, 
% \src{Act05}{BookMain.java}, 
% \src{Act05}{Book.java}, 
% \src{Act05}{Kid.java}
}

% \model{Console Input using Scanner}

Java provides the \java{Scanner} class 
for reading user input from the console, a.k.a.~command line. 
Review and run \emph{src/consoleInput/ConsoleInputMain.java}. 
Then, use \emph{ConsoleInputMain.java} and 
the Java API docs for the \java{Scanner} class 
to answer the following questions. 

\quest{15 min}

\Q What type of object does the \java{Scanner} constructor take as a parameter? 
In this example, what is the provided argument? 
To which very familiar Java feature does it connect? 
\begin{answer}[4em]
Here, the \java{Scanner} constructor takes an \java{InputStream} object 
named \java{System.in}. 
This is the input counterpart to the familiar \java{System.out}, 
which is the \java{PrintStream} object we use to print messages to the console. 
\end{answer}

\Q Based on the description at the top of the Java API docs for \java{Scanner}, 
what are some other options for constructing a \java{Scanner} object? 
\begin{answer}[2em]
We can also construct a Scanner that reads from a File or String. 
\end{answer}

\Q\label{scannerTokens} The example uses the \java{next()} method to read String inputs. 
Based on the Java API docs, how does \java{next()} know when to stop? 
\begin{answer}[3em]
The \java{next()} method reads the next \emph{complete token}, 
which is a sequence of characters between \emph{delimiters} 
(by default, whitespace characters, but they could be commas, semicolons, etc.). 
\end{answer}

\Q The example uses \java{nextInt()} and \java{nextDouble()} 
to read integer and floating-point inputs. 
What type of exception do you get if you try entering 
an invalid \java{int}/\java{double} value? 
\begin{answer}[2em]
\java{InputMismatchException}
\end{answer}

\Q Test out the alternate approaches for reading \java{age} and \java{height}. 
These use static methods from the \java{Integer} and \java{Double} wrapper classes 
to convert the \java{String} returned by \java{scanner.nextLine()} into the appropriate type.  
What happens if we comment out the \java{scanner.nextLine();} call before reading the age? 
\begin{answer}[3em]
We get a \java{NumberFormatException} after entering the name. 
The two calls to \java{next()} did not consume the newline character, \verb|\n|, 
so we passed an empty string to \java{Integer.parseInt}. 
\end{answer}


% \newpage
% \model{Command-Line Interface}

We can use Java's \java{Scanner} to implement simple 
command-line interfaces (CLIs) for our applications. 
In \emph{src/animalShelter}, 
our familiar \java{AnimalShelterMain} class now has a CLI. 
Run the program to try out the CLI, 
then answer the following questions. 

\quest{20 min}

\Q How does the CLI allow for an unlimited number of user commands? 
\begin{answer}[2em]
A \java{while} loop processes one command at a time, 
quitting only when it receives the ``quit'' command. 
\end{answer}

\Q Try adding extra whitespace at the beginning and end of commands 
and mixing up the case (i.e., capitalization). What do you notice? 
What in the code explains your observations? 
\begin{answer}
The CLI ignores extra leading/trailing whitespace, and 
can handle any mix of uppercase/lowercase letters. 
This is accomplished using the \java{trim()} and \java{toLowerCase()} methods 
from the \java{String} class on the result of \java{scanner.nextLine()} 
when reading the next command. 
\end{answer}

\Q Examine the \java{handleCommand} method. 
For the \texttt{add animal} command, 
how does it avoid storing a blank name or species? 
\begin{answer}[2em]
It checks the trimmed input String and 
replaces it with \texttt{``UNKNOWN''} if empty. 
\end{answer}

\Q\label{removeAnimal} In pairs, add a new command called \texttt{``remove animal''} that:
\begin{itemize}
\item prompts the user for an animal ID, 
\item verifies that an animal with that ID exists, 
\item prompts the user to confirm the removal with 
    a ``y'' for yes or ``n''/other for no, and 
\item removes the animal from the shelter records. 
\end{itemize}
When you have finished implementing and testing your remove animal command, 
compare your solution with that of the other pair in your team. 
\begin{answer}
See solution copy of \emph{AnimalShelterMain.java}. 
\end{answer}

% \newpage
% \model{Prototyping Designs (with GenAI)}

Sometimes, while we are learning object-oriented design principles, 
the flaws in a design will not be immediately obvious to us from the UML. 
In this activity, we see how rapid prototyping with the help of GenAI tools 
can help us make connections between design flaws in UML and 
how those flaws manifest in Java code. 

\subsection*{Demo (15 min)}

Recall the Book Tracker Problem from a previous class. 
\begin{quote}
A website tracks books and the kids that read them. 
For each book the system stores the name and author. 
For each kid the system stores name and grade level. 
The teacher enters when a kid reads a particular book. 
It should be possible to print a report on a book that includes 
all kids who have read a particular book (with their grade level). 
It should be possible to print a report on a kid that includes 
the books (with authors) a particular kid has read. 
\end{quote}

We saw the following design (Fig.~\ref{fig:bookTrackerBadA}) 
does not function correctly (violating principle 1 a/b), 
because there is no sane way to look up a book for printing a report 
or associating with a kid (when handling new reading). 

\begin{figure}[!ht]
\centering
\includegraphics[width=0.7\textwidth]{bookTrackerBadA.png}
\caption{Bad Design A for the Book Tracker Problem (\href{https://www.plantuml.com/plantuml/uml/VP3HJeKm38RlznGDLs3q3OmXNdYZ4iCRLDh0mjX8DpJ6yEvkWu3dufoRjca\_\_lzdjGyeOPcCygEsCmfEuCEFuLX5T\_cYEeFUmxDp8\_mg0B0uSUguXC59ccRHDhGyEmbbbKz4hvhAKLEbrbqWBYw\_xmM9t\_YBpJMDv5b-nf9K\_kLuHsqthG4j6MxuksKaRVjC9uDxYEPD7AeJ-pWjNfpi0MyoHu2DMbNwYUAkWCU7fzmJe8v8-p92gcPB-VN-0000}{view PlantUML})}%
\label{fig:bookTrackerBadA}
\end{figure}

\Q The \texttt{src/bookTrackerBadA} folder shows 
the code resulting from Bad Design A. 
Review the code, and try running it. 
What issues do you notice? 
\begin{answer}[6em]
The \java{handleNewReading} method creates 
a new \java{Book} object (with author ``Unknown'') every time 
since \java{BookMain} can't access existing books directly. 
The \java{handlePrintBookNames()} method reveals  
we may accidentally create multiple books with the same name. 
The \java{handlePrintReportForBook} method has to 
look through every kid's books to try to find the target book, 
which is very indirect and confusing. 
\end{answer}

We saw the following improved design (Fig.~\ref{fig:bookTrackerImproved}). 
To see why this design is better, 
let's have a GenAI tool refactor our code 
from Bad Design A to this design. 

\begin{figure}[!ht]
\centering
\includegraphics[width=0.7\textwidth]{bookTrackerImproved.png}
\caption{Improved Design for the Book Tracker Problem (\href{https://www.plantuml.com/plantuml/uml/XP31Je0m38RlUug64nXz0sE81oygCRn1fGqiZ8rqGsECx-uw2MYCyT8sxSV\_\_woD2ILjYpDnjdv5mGLYUdESgzYXNCpWC4Qu3M66TmC07XVMZbkQWkYqYlMfvpL8gfjo8hgtLC-M6lEn2-J5p-z7GU87Vc7tbuPwri-vgCgt78ze\_PVMX9uST\_pQCvBrOw7Lu1AoUPU7viIUJekx81hmF-O8e9I6q9-eBYgulhefNG5pHDgN6VgrAs3BdlAJVW80}{view PlantUML})}%
\label{fig:bookTrackerImproved}
\end{figure}

\Q Take notes here on how we prompted the GenAI tool. 
\begin{answer}
See \href{https://claude.ai/share/619556c4-f49c-48db-80e5-0a347f0b6f3b}{sample Claude.AI conversation}. 
Prompts should clearly ask for refactoring, give the old design and code, 
and give the new design. Be sure to mention the designs are in PlantUML. 
\end{answer}

\Q\label{bookMainDiffs} Compare the \emph{BookMain.java} version from Bad Design A with 
the \emph{BookMain.java} version produced by the GenAI tool. 
What primary differences do you observe? 
\begin{answer}[6em]
The revised \emph{BookMain.java} should add a field 
(\java{ArrayList} or \java{HashMap}) 
for storing all \java{Book} objects. 
If still using \java{ArrayList} fields, 
it may also have new helper methods such as 
\java{findBookByName} and \java{findKidByName}, 
which loop through the lists to search. 
All ``handle'' methods should be simpler except 
\java{handlePrintKidNames}, which was fine to begin with. 
\end{answer}
% \model{Encapsulation}

We end today with one of the pillars of object-oriented software design: 
\emph{encapsulation}. 

\quest{30 min}

\Q\label{functionalButBad} 
If our software system meets all requirements and has no bugs, 
in what ways could it still be poorly designed? 
\begin{answer}[3em]
The code may be hard to read/understand. 
It may also be difficult to add/modify features. 
\end{answer}

\Q Suppose Java's \java{Math} and \java{String} classes 
were merged into one class, putting methods like \java{sqrt()} and 
\java{substring()} next to each other. 
What would be some negative consequences of this merger? 
\begin{answer}[6em]
It would be harder to know what is meant by a method like ``add''  or ``length''. 
And when using/importing a method from the old Math or String class, 
you would now load all of the other class's content too, more than you might actually need. 
It would also be harder to find what you're looking for in the Java API docs for the merged class. 
\end{answer}

\Q In Java, 
it is possible to put all variables and methods for a system in one big class. 
Why is this a bad idea? 
\begin{answer}[5em]
Possible answers: 
It will be hard to find variables and methods we are looking for. 
Lots of unrelated things will be thrown together 
(e.g., similar-named variables/methods 
that would normally be in different classes), 
which can be confusing. 
The system may be less secure since all methods can access all fields, 
not just the ones they need to access. 
\end{answer}

\Q Design Principle 3 (OODP 3) states, ``Functionality should be distributed efficiently:
(a) No class/part should get too large, and (b) Each class should have a single responsibility.'' 
As a team, restate this principle in your own words, 
using simpler language. 
\begin{answer}
Possible answer: 
Jobs (and data) should be spread out among right-sized classes, each with one job. 
\end{answer}

\Q \textbf{Constellation Design Problem:} 
A particular program is designed to load constellations from datafiles 
and draw them on the screen. 
The datafiles include the names of the constellations and 
details about star location, size, and color as well as 
which stars ought to be connected to draw the constellation. 
Depending on the star data, each star should be drawn differently 
(e.g., right size, right color). 

Apply our first three object-oriented design principles 
to evaluate the following design. 

\begin{center}
\includegraphics[width=0.7\textwidth]{constellationBad.png}

(\href{https://www.plantuml.com/plantuml/uml/PP1HIWGn38RVUugmJxioNa5MTl5f5M5E4DhgXisafQd8ADxTjam71rzQaZ\_9\_oL7pPZonyH6RoGH4qwWUMPVtaIsASOogi9ucv-H1BuDm1N5iRy4T6VC-4RizpQ8Pi-CcOAyuEG7S8jMeyFQTIQDZFCZy-blY-xV4yOhMJsOd-wvgMZEKcQLZqcybh\_YlmQbglVu53Ya7epqrGeve7kgd1L85Y2Ny7DZiJ9qzL9z\_cN7CdXD3X0NW05ijHv0YsdRej\_i\_c5tjzjkOypHYolt\_GK0}%
{PlantUML source})
\end{center}

\begin{answer}
This design violates Design Principle 3a, because 
the \java{Constellation} class does everything except maybe 
some input parsing handled in \java{SkyMain}. 
\end{answer}

\Q How could we improve this Constellation design? 
\begin{answer}
Create a \java{Star} class that stores position, color, and size, and 
has a \java{drawStar} method. 
See \href{https://www.plantuml.com/plantuml/uml/RP3HIaCn38Rl-nHbLvleAyZ4uTNqPa\_mqqONbgQ7faEc-EwU7Zgru4tJaF\_BLxgrYbBFIPoTM2SK96\_r8ZIVXSCoSK5WvW-dopDO\_QVp\_WYDGlkCk4F5AmkjGrQh98BAML-GwDR7FclTveVQiKs2ouF8ur\_4rcy5qv63RTpNrJaa5hCYqLo4bQpLZFZKzCsZtHCBtWVoTtMVxXl-d\_1GKHRFb8qREbz3bbpcQln1dM-n4R\_-pjtzwcOrljiDNK-q3MvB6iz9lW40}{revised PlantUML}. 
This is an example of a general pattern: 
Often, we will need to \emph{extract} a new class when one gets too complex. 
(Here, \java{Star} was extracted from \java{Constellation}.) 
\end{answer}


\textbf{Interlude: Defining Encapsulation}

\emph{Encapsulation} refers to the bundling (in a class) of \emph{related data} 
with \emph{methods that operate on that data}. 
A well-designed class forms a sort of ``capsule'' that has a single job and does it well. 
We can pass around the capsule (as object references), and 
other classes can use the methods to get things done 
without needing to know (or interfering with) the private data inside. 

Another benefit of encapsulation is that 
you can often change \emph{how} a class works on the inside 
without needing to change any classes that use that class, 
as long as the public methods stay the same. 
For example, if someone comes up with a better %fancy new-and-improved 
way of implementing a HashMap, 
Java can change the \java{HashMap} class internals 
without breaking existing Java code. 

\Q\label{encapsAnalogy} Come up with a real-world analogy 
    that helps explain the benefits of encapsulation. 
\begin{answer}[5em]
Possible answers: 
Iceberg, where the tip is the public methods of a class and 
the underwater portion is the implementation details. 
A sports team with specialized roles/positions, 
where each player is like a class with a single job. 
Players work together, but they don't need to think about every little detail 
that their teammates are thinking about to perform their roles. 
\end{answer}

\Q\label{pizza} \textbf{Pizza Restaurant Design Problem:} 
A pizza restaurant needs to calculate the cost of orders 
and record what pizzas need to be made. 
An order consists of a number of pizzas, 
each of which might have toppings, 
as well as a customer's name and an order date. 
Each pizza costs \$8 with no toppings. 
The first 2 toppings cost \$2 apiece. 
Each additional topping beyond that costs \$1. 
If a pizza has just peppers, onions, and sausage, 
that's ``The Special'' and it costs \$12. 

\begin{enumerate}
\item What classes do you think are needed? \hfill 
\ans{\java{PizzaMain}, \java{Order}, \java{Pizza}}
\item Clearly, we need a method to computes a pizza's cost. 
Where should it go? Why? 
\begin{answer}[1em] 
\java{Pizza}, because it has the needed data. 
And \java{Order} already has its own job. 
\end{answer}
\item Compare 
\href{https://www.plantuml.com/plantuml/uml/NOzBReCm48RtFeN52ghs2XKLNROaYrnWX4V4YX\_8CrOKejmzC844PVFBZs--NrCJG-BidQAJ3GCay9huud3AP3lvKPq38hstumZ\_O8E-AgsF48p39Y4mjdZU9OEfx39nz9X-WQ5wGZ\_6o6mPPwZD\_bDp70ORUghKRTMBGTGhDXstmkaAY-aG7tsBviyIptKzSXE9ourLSB6kbU\_SQwslx-AZq69MaikB4Ah6OB9tTm00}%
{Solution A} 
and 
\href{https://www.plantuml.com/plantuml/uml/JOzDJiGm34RtEKKqgnR15T2W8bPGMFG2LcELYFn4jYD44NUdILhenehZb\_SvLnOWoTuf\_h0X0O7NB5yEjqfshXCrEs3MRtPTuGLiqDzAwtS8nk50288ZVhwIGUhcp18zqYC8z7\_GWp7rRHSBD6P\_goMcPCF2lVevzDLGrGUs7S\_2hONTDCMMjskUBKjVwEXJ5XmYorEabfbABSC5fGowKsGbQkInJ-EclMli\_x\_lxYytbxQeghNTD8-wOZ3Pkry0}%
{Solution B}. 
Which is a better design, and why? 
\begin{answer}[5em]
Solution A is better, 
because it distributes functionality by 
giving \java{Pizza} a meaningful \java{getCost()} method. 
In Solution B, 
\java{Pizza} is just a ``dumb'' or ``inert'' data holder. 
(Aside: Moving methods like this 
from the \emph{has-a} class to the \emph{has'ed} class 
is sometimes called \emph{pushing functionality down}.)
\end{answer}

\item \emph{For this part and the next one, 
split into pairs (unless you only have three today). 
Each pair should choose a \underline{different} genAI tool.} 

Use a GenAI tool to implement both Solution A and Solution B 
(see~\ref{prototyping-genai.tex}). 
Prompt it to highlight the differences in the two implementations. 
Also ask it to evaluate the solutions with respect to OODP 3. 
\begin{answer}[3em]
See \href{https://claude.ai/share/8bbe0b04-1db2-4e15-91f7-03e1fd8eebac}%
{sample Claude.AI conversation}. 
\end{answer}
\item Did the GenAI tool faithfully implement each design, 
or did it change some things without telling you? 
If something changed, what do you think was the motivation? 
(Feel free to use follow-up prompts.) 
\begin{answer}[3em]
Answers will vary. You should note any fields or methods 
that the genAI tool added, changed, or removed relative to the provided UML. 
\end{answer}
\item Did the genAI tool's assessment of the designs 
match yours from (c)? Explain. 
\begin{answer}[4em]
Answers will vary. 
\end{answer}
\item \emph{Merge your pairs back together and discuss as a full team.} 
Which genAI tool gave the better implementation and design comparison? Explain. 
\begin{answer}[5em]
Answers will vary. 
\end{answer}
\item We could continue extracting classes 
(e.g., \java{Customer}, \java{Topping}). 
Should we? Explain. 
\begin{answer}[5em]
It depends on whether there are other system requirements (perhaps coming soon).
Based on the current system description, 
these classes would just be dumb data holders. 
But if we anticipate other reqs such as 
a customer rewards program or varied topping prices, 
we might want to preemptively add these classes. 
\end{answer}
\end{enumerate}



\end{document}
