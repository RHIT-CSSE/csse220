\model{HashMaps}
% based on Maps slides/activities from CSSE 220: ClassMaterials/MapsAndObjectIntro
% See: Part3-ObjectsAsArrayTypes.pptx, Part4-Maps.pptx

% TODOs: 
% - Create (as starter code) AnimalShelterMain.java that stores Animal objects in an ArrayList. 
% - Have student teams discuss disadvantages of ArrayList for quickly searching for an animal by ID or name. 
% - Refactor (live coding) to use a HashMap instead. 
% - Add HashMap-based implementation to solutions folder. 

\quest{25 min}

We will often find useful a \emph{map} data structure, 
which associates \emph{keys} with \emph{values}. 
A key is some type of unique identifier, and 
a value is an object containing the data attached to that identifier. 
Each key-value pair is called an \emph{entry} in the map. 
In some languages, a map is called a dictionary; 
keys are like words in a dictionary, and 
the values are like definitions. 

\Q Suppose we have a \java{Student} class and we want to create a map 
that lets us look up a student by their (String) username. 
Which should be the key type, and which should be the value? 
\begin{answer}[2em]
The keys should be Strings (the usernames), and 
the values should be the \java{Student} objects. 
\end{answer}

The built-in Java map class that we will use is \java{HashMap}. 
Behind the scenes, \java{HashMap} uses a technique called hashing 
to speed up its operations. 
You will learn more about hashing in CSSE 230, 
but for now, just know that HashMap lookups are fast. 
We declare/initialize HashMaps like we do for ArrayLists, but 
with two type parameters (key, then value), instead of one. 

\Q Complete the following declaration and initialization 
of the student-username map. 

\java{HashMap<}\ans[6em]{\java{String}}, 
\ans[6em]{\java{Student}}\java{> usernameToStudent = new HashMap<>();}

\Q Find the HashMap class docs in the Java API. 
Looking at its methods, how would we\dots

add the new entry \java{username} $\to$ \java{student}? 
\hfill
\ans[22em]{\java{usernameToStudent.put(username, student);}}

check if the map already has \java{username}? 
\hfill
\ans[22em]{\java{usernameToStudent.containsKey(username)}}

check if the map already has \java{student}? 
\hfill
\ans[22em]{\java{usernameToStudent.containsValue(student)}}

determine whether the map has no entries? 
\hfill
\ans[22em]{\java{usernameToStudent.isEmpty()}}

\note{Alternative: use \java{size()} and compare to zero. 
But \java{isEmpty()} is preferred.}

% remove all entries from the map? 
% \hfill
% \java{usernameToStudent.clear();}

retrieve the Student for \java{username}, if it exists? 
\hfill
\ans[22em]{\java{usernameToStudent.get(username);}}

\note{There is also \java{getOrDefault} 
if you don't want to return \java{null} on failed lookups.}

You will encounter other useful HashMap methods 
in a programming assignment. 
We'll now see how HashMaps can accelerate object lookups 
based on identifiers. 

\Q Review \textit{AnimalShelterMain.java}. 
How does this class currently store its \java{Animal} objects? 

\begin{answer}[1.2em]
It uses an \java{ArrayList} of type \java{Animal} named \java{animals}. 
\end{answer}


\Q Examine the methods \java{getAnimalbyName} \& 
\java{updateAnimalWeight}. 
What flaw(s) do you see? 
\begin{answer}
To find an Animal given its name or ID, 
these methods iterate through the entire ArrayList of animals. 
This could take a long time if the animal we're looking for is near the end. 
\end{answer}

\Q Replace the \java{animals} field of type \java{ArrayList<Animal>} 
with two HashMap fields: \java{animalsByID} and \java{animalsByName}. 
What should the key and value types be for these HashMaps? 
\begin{answer}
\java{animalsByID} maps IDs to Animals, 
so its declaration should be 
\java{HashMap<Integer, Animal>}. 

\java{animalsByName} maps names to Animals, 
so its declaration should be 
\java{HashMap<String, Animal>}. 
\end{answer}

\Q\label{mapRefactor} Update the constructor, 
\java{addAnimal}, \java{getAnimalbyName}, and 
\java{updateAnimalWeight} methods to use the new fields. 
After all changes, run the program again. 
How has the output changed? 
\begin{answer}[5em]
The HashMap is printed a bit differently from the old ArrayList 
due to their \java{toString} implementations. 
With the HashMap, when we add an animal with a repeated ID, 
it overwrites the old entry in \java{animalsByID} with that key. 
\end{answer}

\Q What is one potential downside to using the second \java{HashMap}, 
\java{animalsByName}? 
\begin{answer}
The shelter may get animals with matching names; 
the \java{animalsByName} map will only store the most recently added animal. 
\end{answer}
