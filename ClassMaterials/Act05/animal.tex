\model{Shelter Animal}
% based on Model 2 of "Activity 10 - Class Design" by Helen Hu

Classes often represent objects in the real world.
In this section, you will design a new class 
that represents an \java{Animal} in a shelter like the one below. 
This will be the first in a series of activities 
in which we build up a record-keeping system for an animal shelter. 

\begin{center}
\includegraphics[width=0.6\textwidth]{shelter-cat.jpg}

{
\footnotesize%
Photo by \href{https://unsplash.com/@ankumpan?utm\_content=creditCopyText&utm\_medium=referral&utm\_source=unsplash}{Anna Kumpan} 
on \href{https://unsplash.com/photos/white-and-black-cat-on-white-metal-frame-Pm-8aIzBtvo?utm\_content=creditCopyText&utm\_medium=referral&utm\_source=unsplash}{Unsplash}
}
\end{center}


\quest{15 min}


\Q Identify four or more attributes (a.k.a.~fields) 
that should be in the \java{Animal} class.
For each attribute, indicate what data type would be most appropriate.

\begin{answer}
Answers may include
\verb|name:String|, 
\verb|id:int/String|, 
\verb|species:String|, 
\verb|ageYears:int|,
\verb|dateArrived:Date|, 
\verb|birthdate:Date|,  
\verb|sex:String|, 
\verb|size:String|, 
\verb|weight:double|, 
\verb|color:Color|, 
\verb|isSpayedNeutered:boolean|, 
\verb|isHouseTrained:boolean|, etc.
\end{answer}


\Q Using UML syntax, define two or more constructors for the \java{Animal} class.

\begin{answer}
\begin{javaans}
+Animal(id:int)
+Animal(id:int, name:String, species:String)
\end{javaans}
\end{answer}


\Q Define two or more accessor methods for the \java{Animal} class.
Include arguments and return values, using the same format as a UML diagram.

\begin{answer}[5em]
\begin{verbatim}
+getID(): int
+getName(): String
+getSpecies(): String
+getAgeYears(): int
+getWeight(): double
\end{verbatim}
\end{answer}


\Q Define two or more mutator methods for the \java{Animal} class.
Include arguments and return values, using the same format as a UML diagram.

\begin{answer}[5em]
\begin{verbatim}
+setWeight(weight:double): void
+setName(name:String): void
+setSpayedNeutered(status:boolean): void // could omit parameter
+setAgeYears(age:int): void
\end{verbatim}
\end{answer}


\Q \label{key3}
Describe how you would implement the \java{equals} method of the \java{Animal} class.

\begin{answer}
Two animals would be considered equal if they have the same ID number, 
assuming we have been careful to keep these IDs unique. 
\end{answer}


\Q Describe how you would implement the \java{toString} method of 
the \java{Animal} class. 

\begin{answer}
The \java{toString} would print the animal's ID, species, name, 
and other field values, each separated by a comma.
\end{answer}


\Q When constructing (or updating) an \java{Animal} object, 
which arguments would you need to validate? 
What are the valid ranges of values for each attribute? 

\begin{answer}[5em]
The ID number should have a consistent length (number of digits), 
age in years should be nonnegative, 
dates need to have valid months and days, 
names should be at most some upper bound number of letters and 
not contain digits or other characters, 
weight should be positive, etc. 
\end{answer}
