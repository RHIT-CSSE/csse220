% comment out for student version
\ifdefined\Student\relax\else\def\Teacher{}\fi

\documentclass[12pt]{article}

\title{Designing Objects}
\author{Ian Ludden}
\date{Summer 2025}

\input{../../cspogil.sty}

\begin{document}

\maketitle

Today, we look at how to create and apply custom object types in Java. 
We will also see a useful data structure called a HashMap.

\ifdefined\Student\rolenames{}\fi

\guide{
    \item I can explain the purpose of constructor, accessor, and mutator methods. 
    \item I can implement the equals and toString methods for a given class design. 
    \item I can design a new class (UML diagram) based on a general description. 
    \item I can explain what a HashMap represents. 
    \item I can list and apply essential HashMap operations. 
    \item I can describe the benefits and limitations of HashMaps. 
}{
    \item Identifying attributes and data types that model a real-world object. (Problem Solving)
}{
\ref{common-methods.tex} begins with questions about constructors, getters, and setters.
Report out as soon as the majority of teams have finished. 
You may find it helpful to project the source code for \textit{Color.java} 
while students work through the questions. 
Reinforce the concept of immutable objects, 
and point out that the \java{String} class is designed this way. 

The questions at the beginning of \ref{equals-tostring.tex} 
require students to understand the source code of \java{equals} and \java{toString}. 
If this is the first time they have seen language features like \java{Object}, 
\texttt{instanceof}, and \java{String.format}, 
you might want to give a 5-minute lecture (but avoid giving away the answers). 

When reporting out \ref{animal.tex}, have presenters write their designs on the board. 
Compare the trade-offs of their different designs.
For example, to store ID numbers some teams may use strings, others may use ints/longs/arrays of ints. 

\ref{maps.tex} introduces HashMaps in the context of the Animal Shelter example. 

Key questions: \ref{key1}, \ref{key2}, \ref{key3}, \ref{mapRefactor}

Source files: 
\src{Act05}{Color.java}, 
\src{Act05}{Point.java}, 
\src{Act05}{AnimalShelterMain.java}
}

\input{common-methods.tex}
\newpage
\input{equals-tostring.tex}
\vspace{1em}
\model{Shelter Animal}
% based on Model 2 of "Activity 10 - Class Design" by Helen Hu

Classes often represent objects in the real world.
In this section, you will design a new class 
that represents an \java{Animal} in a shelter like the one below. 
This will be the first in a series of activities 
in which we build up a record-keeping system for an animal shelter. 

\begin{center}
\includegraphics[width=0.6\textwidth]{shelter-cat.jpg}

{
\footnotesize%
Photo by \href{https://unsplash.com/@ankumpan?utm\_content=creditCopyText&utm\_medium=referral&utm\_source=unsplash}{Anna Kumpan} 
on \href{https://unsplash.com/photos/white-and-black-cat-on-white-metal-frame-Pm-8aIzBtvo?utm\_content=creditCopyText&utm\_medium=referral&utm\_source=unsplash}{Unsplash}
}
\end{center}


\quest{15 min}


\Q Identify four or more attributes (a.k.a.~fields) 
that should be in the \java{Animal} class.
For each attribute, indicate what data type would be most appropriate.

\begin{answer}
Answers may include
\verb|name:String|, 
\verb|id:int/String|, 
\verb|species:String|, 
\verb|ageYears:int|,
\verb|dateArrived:Date|, 
\verb|birthdate:Date|,  
\verb|sex:String|, 
\verb|size:String|, 
\verb|weight:double|, 
\verb|color:Color|, 
\verb|isSpayedNeutered:boolean|, 
\verb|isHouseTrained:boolean|, etc.
\end{answer}


\Q Using UML syntax, define two or more constructors for the \java{Animal} class.

\begin{answer}
\begin{javaans}
+Animal(id:int)
+Animal(id:int, name:String, species:String)
\end{javaans}
\end{answer}


\Q Define two or more accessor methods for the \java{Animal} class.
Include arguments and return values, using the same format as a UML diagram.

\begin{answer}[5em]
\begin{verbatim}
+getID(): int
+getName(): String
+getSpecies(): String
+getAgeYears(): int
+getWeight(): double
\end{verbatim}
\end{answer}


\Q Define two or more mutator methods for the \java{Animal} class.
Include arguments and return values, using the same format as a UML diagram.

\begin{answer}[5em]
\begin{verbatim}
+setWeight(weight:double): void
+setName(name:String): void
+setSpayedNeutered(status:boolean): void // could omit parameter
+setAgeYears(age:int): void
\end{verbatim}
\end{answer}


\Q \label{key3}
Describe how you would implement the \java{equals} method of the \java{Animal} class.

\begin{answer}
Two animals would be considered equal if they have the same ID number, 
assuming we have been careful to keep these IDs unique. 
\end{answer}


\Q Describe how you would implement the \java{toString} method of 
the \java{Animal} class. 

\begin{answer}
The \java{toString} would print the animal's ID, species, name, 
and other field values, each separated by a comma.
\end{answer}


\Q When constructing (or updating) an \java{Animal} object, 
which arguments would you need to validate? 
What are the valid ranges of values for each attribute? 

\begin{answer}[5em]
The ID number should have a consistent length (number of digits), 
age in years should be nonnegative, 
dates need to have valid months and days, 
names should be at most some upper bound number of letters and 
not contain digits or other characters, 
weight should be positive, etc. 
\end{answer}

\newpage
\model{HashMaps}
% based on Maps slides/activities from CSSE 220: ClassMaterials/MapsAndObjectIntro
% See: Part3-ObjectsAsArrayTypes.pptx, Part4-Maps.pptx

% TODOs: 
% - Create (as starter code) AnimalShelterMain.java that stores Animal objects in an ArrayList. 
% - Have student teams discuss disadvantages of ArrayList for quickly searching for an animal by ID or name. 
% - Refactor (live coding) to use a HashMap instead. 
% - Add HashMap-based implementation to solutions folder. 

\quest{25 min}

We will often find useful a \emph{map} data structure, 
which associates \emph{keys} with \emph{values}. 
A key is some type of unique identifier, and 
a value is an object containing the data attached to that identifier. 
Each key-value pair is called an \emph{entry} in the map. 
In some languages, a map is called a dictionary; 
keys are like words in a dictionary, and 
the values are like definitions. 

\Q Suppose we have a \java{Student} class and we want to create a map 
that lets us look up a student by their (String) username. 
Which should be the key type, and which should be the value? 
\begin{answer}[2em]
The keys should be Strings (the usernames), and 
the values should be the \java{Student} objects. 
\end{answer}

The built-in Java map class that we will use is \java{HashMap}. 
Behind the scenes, \java{HashMap} uses a technique called hashing 
to speed up its operations. 
You will learn more about hashing in CSSE 230, 
but for now, just know that HashMap lookups are fast. 
We declare/initialize HashMaps like we do for ArrayLists, but 
with two type parameters (key, then value), instead of one. 

\Q Complete the following declaration and initialization 
of the student-username map. 

\java{HashMap<}\ans[6em]{\java{String}}, 
\ans[6em]{\java{Student}}\java{> usernameToStudent = new HashMap<>();}

\Q Find the HashMap class docs in the Java API. 
Looking at its methods, how would we\dots

add the new entry \java{username} $\to$ \java{student}? 
\hfill
\ans[22em]{\java{usernameToStudent.put(username, student);}}

check if the map already has \java{username}? 
\hfill
\ans[22em]{\java{usernameToStudent.containsKey(username)}}

check if the map already has \java{student}? 
\hfill
\ans[22em]{\java{usernameToStudent.containsValue(student)}}

determine whether the map has no entries? 
\hfill
\ans[22em]{\java{usernameToStudent.isEmpty()}}

\note{Alternative: use \java{size()} and compare to zero. 
But \java{isEmpty()} is preferred.}

% remove all entries from the map? 
% \hfill
% \java{usernameToStudent.clear();}

retrieve the Student for \java{username}, if it exists? 
\hfill
\ans[22em]{\java{usernameToStudent.get(username);}}

\note{There is also \java{getOrDefault} 
if you don't want to return \java{null} on failed lookups.}

You will encounter other useful HashMap methods 
in a programming assignment. 
We'll now see how HashMaps can accelerate object lookups 
based on identifiers. 

\Q Review \textit{AnimalShelterMain.java}. 
How does this class currently store its \java{Animal} objects? 

\begin{answer}[1.2em]
It uses an \java{ArrayList} of type \java{Animal} named \java{animals}. 
\end{answer}


\Q Examine the methods \java{getAnimalbyName} \& 
\java{updateAnimalWeight}. 
What flaw(s) do you see? 
\begin{answer}
To find an Animal given its name or ID, 
these methods iterate through the entire ArrayList of animals. 
This could take a long time if the animal we're looking for is near the end. 
\end{answer}

\Q Replace the \java{animals} field of type \java{ArrayList<Animal>} 
with two HashMap fields: \java{animalsByID} and \java{animalsByName}. 
What should the key and value types be for these HashMaps? 
\begin{answer}
\java{animalsByID} maps IDs to Animals, 
so its declaration should be 
\java{HashMap<Integer, Animal>}. 

\java{animalsByName} maps names to Animals, 
so its declaration should be 
\java{HashMap<String, Animal>}. 
\end{answer}

\Q\label{mapRefactor} Update the constructor, 
\java{addAnimal}, \java{getAnimalbyName}, and 
\java{updateAnimalWeight} methods to use the new fields. 
After all changes, run the program again. 
How has the output changed? 
\begin{answer}[5em]
The HashMap is printed a bit differently from the old ArrayList 
due to their \java{toString} implementations. 
With the HashMap, when we add an animal with a repeated ID, 
it overwrites the old entry in \java{animalsByID} with that key. 
\end{answer}

\Q What is one potential downside to using the second \java{HashMap}, 
\java{animalsByName}? 
\begin{answer}
The shelter may get animals with matching names; 
the \java{animalsByName} map will only store the most recently added animal. 
\end{answer}


\end{document}
