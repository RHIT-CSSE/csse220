\model{Command-Line Interface}

We can use Java's \java{Scanner} to implement simple 
command-line interfaces (CLIs) for our applications. 
In \emph{src/animalShelter}, 
our familiar \java{AnimalShelterMain} class now has a CLI. 
Run the program to try out the CLI, 
then answer the following questions. 

\quest{20 min}

\Q How does the CLI allow for an unlimited number of user commands? 
\begin{answer}[2em]
A \java{while} loop processes one command at a time, 
quitting only when it receives the ``quit'' command. 
\end{answer}

\Q Try adding extra whitespace at the beginning and end of commands 
and mixing up the case (i.e., capitalization). What do you notice? 
What in the code explains your observations? 
\begin{answer}
The CLI ignores extra leading/trailing whitespace, and 
can handle any mix of uppercase/lowercase letters. 
This is accomplished using the \java{trim()} and \java{toLowerCase()} methods 
from the \java{String} class on the result of \java{scanner.nextLine()} 
when reading the next command. 
\end{answer}

\Q Examine the \java{handleCommand} method. 
For the \texttt{add animal} command, 
how does it avoid storing a blank name or species? 
\begin{answer}[2em]
It checks the trimmed input String and 
replaces it with \texttt{``UNKNOWN''} if empty. 
\end{answer}

\Q\label{removeAnimal} In pairs, add a new command called \texttt{``remove animal''} that:
\begin{itemize}
\item prompts the user for an animal ID, 
\item verifies that an animal with that ID exists, 
\item prompts the user to confirm the removal with 
    a ``y'' for yes or ``n''/other for no, and 
\item removes the animal from the shelter records. 
\end{itemize}
When you have finished implementing and testing your remove animal command, 
compare your solution with that of the other pair in your team. 
\begin{answer}
See solution copy of \emph{AnimalShelterMain.java}. 
\end{answer}
