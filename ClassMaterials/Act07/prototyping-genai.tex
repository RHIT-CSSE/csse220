\model{Prototyping Designs (with GenAI)}

Sometimes, while we are learning object-oriented design principles, 
the flaws in a design will not be immediately obvious to us from the UML. 
In this activity, we see how rapid prototyping with the help of GenAI tools 
can help us make connections between design flaws in UML and 
how those flaws manifest in Java code. 

\subsection*{Demo (15 min)}

Recall the Book Tracker Problem from a previous class. 
\begin{quote}
A website tracks books and the kids that read them. 
For each book the system stores the name and author. 
For each kid the system stores name and grade level. 
The teacher enters when a kid reads a particular book. 
It should be possible to print a report on a book that includes 
all kids who have read a particular book (with their grade level). 
It should be possible to print a report on a kid that includes 
the books (with authors) a particular kid has read. 
\end{quote}

We saw the following design (Fig.~\ref{fig:bookTrackerBadA}) 
does not function correctly (violating principle 1 a/b), 
because there is no sane way to look up a book for printing a report 
or associating with a kid (when handling new reading). 

\begin{figure}[!ht]
\centering
\includegraphics[width=0.7\textwidth]{bookTrackerBadA.png}
\caption{Bad Design A for the Book Tracker Problem (\href{https://www.plantuml.com/plantuml/uml/VP3HJeKm38RlznGDLs3q3OmXNdYZ4iCRLDh0mjX8DpJ6yEvkWu3dufoRjca\_\_lzdjGyeOPcCygEsCmfEuCEFuLX5T\_cYEeFUmxDp8\_mg0B0uSUguXC59ccRHDhGyEmbbbKz4hvhAKLEbrbqWBYw\_xmM9t\_YBpJMDv5b-nf9K\_kLuHsqthG4j6MxuksKaRVjC9uDxYEPD7AeJ-pWjNfpi0MyoHu2DMbNwYUAkWCU7fzmJe8v8-p92gcPB-VN-0000}{view PlantUML})}%
\label{fig:bookTrackerBadA}
\end{figure}

\Q The \texttt{src/bookTrackerBadA} folder shows 
the code resulting from Bad Design A. 
Review the code, and try running it. 
What issues do you notice? 
\begin{answer}[6em]
The \java{handleNewReading} method creates 
a new \java{Book} object (with author ``Unknown'') every time 
since \java{BookMain} can't access existing books directly. 
The \java{handlePrintBookNames()} method reveals  
we may accidentally create multiple books with the same name. 
The \java{handlePrintReportForBook} method has to 
look through every kid's books to try to find the target book, 
which is very indirect and confusing. 
\end{answer}

We saw the following improved design (Fig.~\ref{fig:bookTrackerImproved}). 
To see why this design is better, 
let's have a GenAI tool refactor our code 
from Bad Design A to this design. 

\begin{figure}[!ht]
\centering
\includegraphics[width=0.7\textwidth]{bookTrackerImproved.png}
\caption{Improved Design for the Book Tracker Problem (\href{https://www.plantuml.com/plantuml/uml/XP31Je0m38RlUug64nXz0sE81oygCRn1fGqiZ8rqGsECx-uw2MYCyT8sxSV\_\_woD2ILjYpDnjdv5mGLYUdESgzYXNCpWC4Qu3M66TmC07XVMZbkQWkYqYlMfvpL8gfjo8hgtLC-M6lEn2-J5p-z7GU87Vc7tbuPwri-vgCgt78ze\_PVMX9uST\_pQCvBrOw7Lu1AoUPU7viIUJekx81hmF-O8e9I6q9-eBYgulhefNG5pHDgN6VgrAs3BdlAJVW80}{view PlantUML})}%
\label{fig:bookTrackerImproved}
\end{figure}

\Q Take notes here on how we prompted the GenAI tool. 
\begin{answer}
See \href{https://claude.ai/share/619556c4-f49c-48db-80e5-0a347f0b6f3b}{sample Claude.AI conversation}. 
Prompts should clearly ask for refactoring, give the old design and code, 
and give the new design. Be sure to mention the designs are in PlantUML. 
\end{answer}

\Q\label{bookMainDiffs} Compare the \emph{BookMain.java} version from Bad Design A with 
the \emph{BookMain.java} version produced by the GenAI tool. 
What primary differences do you observe? 
\begin{answer}[6em]
The revised \emph{BookMain.java} should add a field 
(\java{ArrayList} or \java{HashMap}) 
for storing all \java{Book} objects. 
If still using \java{ArrayList} fields, 
it may also have new helper methods such as 
\java{findBookByName} and \java{findKidByName}, 
which loop through the lists to search. 
All ``handle'' methods should be simpler except 
\java{handlePrintKidNames}, which was fine to begin with. 
\end{answer}