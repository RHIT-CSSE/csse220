% comment out for student version
\ifdefined\Student\relax\else\def\Teacher{}\fi

\documentclass[12pt]{article}

\title{Introduction to Objects and Debugging}
\author{Ian Ludden}
\date{Summer 2025}

\input{../../cspogil.sty}

\begin{document}

\maketitle

Today, we look at how to create and apply custom object types in Java. 

\ifdefined\Student\rolenames{}\fi

\guide{
    \item I can define objects and related terms, such as field, attribute, method, constructor. 
    \item I can identify the fields and methods of a Java object. 
    \item I can create new object types in Java. 
    \item Given a Java class, I can construct the corresponding UML class diagram. 
    \item Given a UML class diagram, I can write the corresponding Java code. 
    \item I can use debugging tools. 
}{
    \item Verifying program behavior by tracing code with a debugger. (Critical Thinking)
}{
The meta activity is ultimately about process skills and should help with student buy-in. 
If you are using the \github{Handouts/role-cards-mayfield.pdf}{Role Cards}, have students look at the definitions on the reverse side. 
Each activity targets specific ``process skill goals'' from these categories. 
Point out that ``Technical skills'' and ``Leadership'' are probably ranked lower than expected. 

In~\ref{die-class.tex}, students analyze a given Java class representing a six-sided die. 

\ref{circle-class.tex} introduces UML class diagrams and how to translate them into code. 

\ref{debugging.tex} introduces the why/what/how of the debugger and ends with debugging practice. 

Key questions: 
\ref{dievar}, \ref{circmain}, \ref{debugOptions}

Source files:
\src{Act04}{Die.java}, 
\src{Act04}{debugger/DebugMe.java}, 
\src{Act04}{debugger/DebugMeTest.java}, 
\src{Act04}{debugger/WhackABug.java}
}

\input{../Meta/employers.tex}
\newpage
\input{die-class.tex} % Replaces BankAccount example
\newpage
\input{circle-class.tex} % Replaces RoseString and SmallClassProbs
\newpage
\model{Debugging}

% Mistakes in source code are called \emph{bugs}. 
% (Why? Outside of class, 
% \href{https://interestingengineering.com/innovation/the-origin-of-the-term-computer-bug}{read more} 
% on this intersection of 
% \href{https://en.wikipedia.org/wiki/Etymology}{etymology} and 
% \href{https://en.wikipedia.org/wiki/Entomology}{entomology}.)
% The process of finding and eliminating software bugs is called \emph{debugging}.

% \begin{center}
% \includegraphics{Circle.pdf}
% \end{center}

\quest{30 min. (+15 min.~demo)}


\Q\label{debugOptions} Discuss (2 min): If a program is not working the way you expect 
(e.g., it is not passing all its tests), 
what are some options for figuring out what's wrong? 
\begin{answer}[4em]
Expected answers: 
change some statements and see what happens, 
print statements to see intermediate variable values, 
read Java error messages (if present), 
look back through code and think through its logic
\end{answer}

A good Integrated Development Environment (IDE) provides a \emph{debugger}, 
a suite of tools for finding and fixing errors in code. 
After the debugger demonstration, 
answer these questions. 

\Q What is the term for a place in which we tell the debugger to pause program execution? 
\ans[6em]{breakpoint}

\Q After the debugger pauses, what primary commands do you have available? 
\begin{answer}[4em]
Resume (to keep going as normal), Terminate (to stop entirely), 
Step Into (to go to the first line of the invoked method, if any), and 
Step Over (to go to the next line in the current method)
\end{answer}

\Q If you get some type of Java Exception 
but aren't sure where or why, what can you do? 
\begin{answer}[3em]
Set an \emph{exception breakpoint} that will pause execution 
whenever the program encounters an exception of the specified type. 
\end{answer}

\Q Open \emph{DebugMe.java} and \emph{DebugMeTest.java}. 
Practice using the debugger to find and fix 
the bug in \java{uppercaseIfExclamation}. 
\begin{answer}[3em]
Java Strings are immutable, 
so calling \java{toUpperCase} does not change \java{sentence}. 
We need to return (or store and return) the modified String. 
\end{answer}

\Q Practice using the debugger to find and fix the bugs 
in \java{kPermutations} and \java{isArrayDoubled}. 
\begin{answer}[3em]
In \java{kPermutations} we encounter an \java{int} overflow. 
In \java{isArrayDoubled} we realize 
the \java{equals} method does not compare arrays' values. 
\end{answer}

\Q Continue debugging practice with \emph{WhackABug.java}. 
Note the bugs and lessons learned. 
\begin{answer}[5em]
See the \emph{WhackABug.java} copy in the solutions folder 
for explanations of the bugs. 
\end{answer}

 % Based on DebugMe and WhackABug

\end{document}
