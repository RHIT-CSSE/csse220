% comment out for student version
\ifdefined\Student\relax\else\def\Teacher{}\fi

\documentclass[12pt]{article}

\title{The \java{static} Keyword and Inheritance}
\author{Ian Ludden}
\date{Summer 2025}

\input{../../cspogil.sty}

\begin{document}

\maketitle

Today, we look at when and how to use the \java{static} keyword in Java. 
We then examine \emph{inheritance}, one of the pillars of object-oriented design.  
\ifdefined\Student\rolenames{}\fi

\guide{
    \item I can differentiate between static and instance variables and methods. 
    \item I can summarize best practices for when to use static variables and methods. 
    \item I can explain what it means for one class to extend another 
        and summarize the \java{extends} and \java{super} keywords. 
    \item I can generalize multiple classes that have overlapping code.
    \item I can explain the requirements of abstract classes and methods. 
    \item I can write a new method for an existing Java library class. 
}{
    \item Reading Java API documentation and making inferences. (Information Processing)
	\item Making conclusions based on IDE hints and program output. (Critical Thinking)
	%\item Identifying similarities and differences between classes and refactoring to use an appropriate superclass. (Critical Thinking)
}{
\textbf{First Hour:}
\ref{static.tex} introduces static fields and methods through examples. 
Be sure to summarize best practices at the end: 
only use static fields for constants (until later courses), and 
only use static methods for utility functions and \java{main}. 

End the first hour with a 10-minute mini lecture that shows 
how to create and add two BigIntegers, and how to find and replace strings. 

\textbf{Second Hour:} 
\ref{extending.tex} introduces inheritance with a custom extension of \java{BigInteger}. 
This activity requires some prior knowledge of BigInteger and the \java{String.replace()} method. 
BigInteger was originally chosen for this activity, because (1) it's a useful class in the Java library, and (2) it's not declared as \java{final} (so it can be extended).
The new feature this activity is adding to BigInteger is the ability to work with comma separators.
% Highlight important aspects of the documentation for BigInteger.
Mention that BigInteger extends Number, which extends Object. 
Notice the static constants. 
% In addition to the constructors, there is a static \java{valueOf} method. 
Prevent students from spending too much time on \ref{count1} and \ref{count2}. 
% You can tell them the number of methods; they should be able to count the fields and constructors. 
% Introduce ``is a'' and ``has a'' terminology when reporting out. 

\ref{abstract.tex} introduces abstract superclasses. 
% Do not provide \textit{LoudToy.java} (the code for Model 2) until after Model 1.
% Students will develop their own version of this class in \ref{LoudToyV1} and later compare it with Model 2.

% \ref{refactoring-inheritance.tex} demonstrates refactoring to use an abstract class 
% in the ongoing Animal Shelter example project. 

Key questions: 
\ref{staticPatterns}, 
\ref{key1}, \ref{key2}, \ref{key3}, 
\ref{abstractKey1}, \ref{abstractKey2}
% \ref{removeAnimal}, 
% \ref{bookMainDiffs}, 
% \ref{functionalButBad}, 
% \ref{encapsAnalogy}, 
% \ref{pizza}

Source files: 
\src{Act08}{Point.java},  
\src{Act08}{MyBigInt.java},  
\src{Act08}{LoudToy.java}, 
\src{Act08}{ToySheep.java}, 
\src{Act08}{ToyRobot.java}, 
\src{Act08}{StudentV1[/V2].java}
}

\model{The \java{static} Keyword}

We have seen the \java{static} Java keyword come up in a few situations. 
It's time for a deep dive into what it means and how to use it. 

{\footnotesize 
\begin{minipage}{0.48\textwidth}
\textbf{Version 1:}
\begin{javalst}
public class StudentV1 {
    private String name;
    private char grade;

    public StudentV1(String name, char grd) {
        this.name = name; 
        this.grade = grd;
    }

    @Override 
    public String toString() {
        return name + " earned " + grade;
    }

    public static void main(String[] args) {
        StudentV1 a = 
            new StudentV1("Adaline", 'A');
        StudentV1 b = 
            new StudentV1("Belicia", 'B');
        StudentV1 c = 
            new StudentV1("Charlie", 'C');
        System.out.println(a);
        System.out.println(b);
        System.out.println(c);
    }
}
\end{javalst}
\end{minipage}\hfill
\begin{minipage}{0.48\textwidth}
\textbf{Version 2:}
\begin{javalst}
public class StudentV2 {
    private String name;
    private static char grade;

    public StudentV2(String name, char grd) {
        this.name = name; 
        StudentV2.grade = grd;
    }

    @Override 
    public String toString() {
        return name + " earned " + grade;
    }

    public static void main(String[] args) {
        StudentV2 a = 
            new StudentV2("Adaline", 'A');
        StudentV2 b = 
            new StudentV2("Belicia", 'B');
        StudentV2 c = 
            new StudentV2("Charlie", 'C');
        System.out.println(a);
        System.out.println(b);
        System.out.println(c);
    }
}
\end{javalst}
\end{minipage}
}

\quest{20 min}

\Q Examine the two versions of a simple \texttt{Student} class above. 
Notice the subtle differences, and predict the output of each. 
\emph{After} predicting, run them (in \texttt{src/student}) to confirm. 
\begin{quote}
\texttt{StudentV1} output: 
\begin{answer}
\begin{javaans}
Adaline earned A
Belicia earned B
Charlie earned C
\end{javaans}
\end{answer}

\texttt{StudentV2} output: 
\begin{answer}
\begin{javaans}
Adaline earned C
Belicia earned C
Charlie earned C
\end{javaans}
\end{answer}
\end{quote}

\Q The English word \emph{static} can take 
\href{https://www.merriam-webster.com/dictionary/static}%
{several different meanings}. 
Based on this example, which meaning do you think Java is using? 
\begin{answer}[3em]
``showing little change'' fits best (or ``stationary''); 
in V2, the static variable \java{grade} is shared 
between all \java{StudentV2} instances. 
Its value is tied to the \emph{class}, 
not individual instances. 
\end{answer}

\Q In the Java API docs for the \java{Integer} class, 
find the list of fields. 
What are some of the \java{static} fields in this class? 
Click on each field's name to see more info. 
\begin{answer}[3em]
\texttt{BYTES}, \texttt{MAX\_VALUE}, \texttt{MIN\_VALUE}, 
\texttt{SIZE}, \texttt{TYPE}. 
Note all are declared as \java{public static final}, 
where the \java{final} keyword means 
their values cannot change once initialized. 
\end{answer}

\Q\label{staticPatterns} Look at \java{static} fields in other Java classes, 
such as \java{Math}, \java{Calendar}, and \java{HttpURLConnection}. 
Based on what you find, 
what do you think are some patterns/conventions/best practices 
for using \java{static} fields? 
\begin{answer}
Notable examples: \java{Math.PI}, 
\java{HttpURLConnection.HTTP_NOT_FOUND}, and 
\java{Calendar.SUNDAY} (and \java{MONDAY}, etc.). 
Patterns: 
(1) Accessed by the syntax \texttt{ClassName.FIELD\_NAME}. 
(2) Usually named with all-caps and underscores. 
(3) Often used to give constants descriptive names. 
\end{answer}

Methods can also be declared as \java{static}, 
meaning they are associated with the class itself and 
not with individual instances. 
In fact, we can call static methods 
(a.k.a.~\href{https://docs.oracle.com/javase/tutorial/java/javaOO/classvars.html}%
{class methods}) without ever creating an instance of the object type. 
A static method has \emph{unchanging} behavior: 
it is not state-dependent, 
meaning it does not do different things depending on 
the data stored in the current object instance. 
Static methods are (almost always) true functions, 
in the mathematical sense: 
for each valid input, they will give the same output every time. 

\Q We have already used one static method in the \java{Integer} class: 
\java{Integer.parseInt(String s)}. 
(Notice the camel-case naming convention and the 
\texttt{ClassName.methodName()} syntax.)
Look at the list of \java{static} methods in \java{Integer}. 
What are some typical use cases for static methods? 
\begin{answer}[2em]
comparing two \java{int} values or finding their max/min, 
counting leading/trailing zeros, 
converting from one numerical representation to another, 
manipulating the binary, etc. 
\end{answer}

\Q We always declare the \java{main} method in Java as \java{static}. 
Why do you think this is necessary? 
\begin{answer}[6em]
If \java{main} was not \java{static}, 
we would have to instantiate an object to call it. 
But we can't run any code (let alone instantiate an object) 
until we start the program by running \java{main}, 
so we would get a 
\href{https://en.wikipedia.org/wiki/Chicken\_or\_the\_egg}%
{chicken-and-egg} problem: 
which came first, the instance or the \java{main} call? 
Java \emph{could} have been designed to start by 
instantiating an object of the main class type and \emph{then} 
calling main, but this can get messy. 
%if our main class has multiple constructors, which should Java call? 
\end{answer}

\Q Examine the provided \emph{Point.java} file. 
Uncomment the print statements, and add the missing methods. 
Which one is static? 
How do the two distance methods differ in ``point'' of view? 
\begin{answer}[5em]
The static method is \texttt{distanceBetween}. 
It represents a ``completely objective third-party observer 
with absolutely no personal interest in the matter'' 
that's looking at \texttt{a} and \texttt{b} 
and measuring the distance in between. 
The non-static (instance) method is \texttt{distanceTo}. 
It represents a ``first-person'' perspective: 
Point \texttt{a} looks at point \texttt{b} and 
measures how far away it is. 
\end{answer}


\newpage
\input{extending.tex}
\newpage
\model{Abstract Classes}

Just like in language and philosophy 
there are abstract ideas and categories that can be realized as concrete examples/things, 
Java allows us to distinguish between \java{abstract} classes and 
\emph{concrete} (the default) classes. 

%%% \model{Loud Toys}

\begin{multicols}{2}
\small

\begin{javalst}
public class ToySheep {
    private int volume;

    public ToySheep() {
        this.volume = 3;
    }

    public int getVolume() {
        return volume;
    }

    public void setVolume(int volume) {
        this.volume = volume;
        makeNoise();
    }

    public void makeNoise() {
        System.out.println("Baaa");
    }
}
\end{javalst}

\begin{center}
\includegraphics[height=8em]{shaun.png}
\hspace{2em}
\includegraphics[height=8em]{wall-e.jpg}
\end{center}

\columnbreak

\begin{javalst}
public class ToyRobot {
    private int chargeLevel;
    private int volume;

    public ToyRobot() {
        this.chargeLevel = 5;
        this.volume = 10;
    }

    public void recharge() {
        chargeLevel = 10;
    }

    public int getVolume() {
        return volume;
    }

    public void setVolume(int volume) {
        this.volume = volume;
        makeNoise();
    }

    public void makeNoise() {
        System.out.println("Beep Beep!");
    }
}
\end{javalst}

\end{multicols}

\quest{25 min}


\Q Identify \emph{similarities} in the code: 
what fields and methods do the classes have in common?
\ans[468pt]{
\java{private int volume},~\java{getVolume()},~\java{setVolume(int volume)},~
\java{makeNoise()}
}
%\\[1ex] \ans[468pt]{volume}
%\\[1ex] \ans[468pt]{getVolume, setVolume, makeNoise}

\Q Summarize \emph{differences} between the two classes. % constructors and \java{makeNoise} methods.

\begin{answer}
Constructor differences: 
ToySheep sets volume to 3, but it's 10 in ToyRobot. 
In \java{makeNoise()}, ToySheep says ``Baaa'', but ToyRobot says ``Beep Beep!''. 
The ToyRobot has an extra \java{recharge()} method. 
\end{answer}

\clearpage

\Q \label{LoudToyV1}
Design a new class named \java{LoudToy} that contains 
the code that \java{ToySheep} and \java{ToyRobots} have in common. 
Its constructor should take \java{volume} as a parameter, and 
\java{makeNoise} should have an empty body.

\begin{javalst}
public class LoudToy {
\end{javalst}
\vspace{-1ex}
\begin{answer}[25em]
\begin{javaans}
    private int volume;

    public LoudToy(int volume) {
        this.volume = volume;
    }

    public int getVolume() {
        return volume;
    }

    public void setVolume(int volume) {
        this.volume = volume;
        makeNoise();
    }

    public void makeNoise() {
        // will be overridden in subclass
    }
\end{javaans}
\end{answer}
\vspace{-1ex}
\begin{javalst}
}
\end{javalst}


\Q Redesign \java{ToySheep} so that it extends \java{LoudToy}.
The constructor of \java{ToySheep} should call the constructor of \java{LoudToy}.
Remove the code from \java{ToySheep} that is no longer necessary.
%Do not duplicate any code in \java{LoudToy}.

\begin{javalst}
public class ToySheep extends LoudToy {
\end{javalst}
\vspace{-1ex}
\begin{answer}[12em]
\begin{javaans}
    public ToySheep() {
        super(3);
    }

    public void makeNoise() {
        System.out.println("Baaa");
    }
\end{javaans}
\end{answer}
\vspace{-1ex}
\begin{javalst}
}
\end{javalst}


\newpage

\Q \label{abstractKey1}
Redesign \java{ToyRobot} so that it extends \java{LoudToy}, and 
remove extraneous code. 
%Do not duplicate any code in \java{LoudToy}. 

\begin{javalst}
public class ToyRobot extends LoudToy {
\end{javalst}
\vspace{-1ex}
\begin{answer}[20em]
\begin{javaans}
    private int chargeLevel;

    public ToyRobot() {
        super(10);
        chargeLevel = 5;
    }

    public void recharge() {
        chargeLevel = 10;
    }

    public void makeNoise() {
        System.out.println("Beep Beep!");
    }
\end{javaans}
\end{answer}
\vspace{-1ex}
\begin{javalst}
}
\end{javalst}


\Q\label{toyOutput} What is the output of the following examples?

\begin{enumerate}

\item
\begin{javalst}
LoudToy toy1 = new LoudToy(1);
toy1.makeNoise();
\end{javalst}

\vspace{-2.5em} \hspace{18em} \ans[20em]{(no output)}

\item
\begin{javalst}
LoudToy toy2 = new ToySheep();
toy2.makeNoise();
\end{javalst}

\vspace{-2.5em} \hspace{18em} \ans[20em]{\texttt{Baaa}}

\item
\begin{javalst}
LoudToy toy3 = new ToyRobot();
toy3.makeNoise();
\end{javalst}

\vspace{-2.5em} \hspace{18em} \ans[20em]{\texttt{Beep Beep!}}

\end{enumerate}

Notice that the \emph{instantiated type} of an object  
can be a subclass of its variable's \emph{declared type}. 
In other words, we can store \java{ObjectType} in a variable with declared type 
\java{DeclaredType} if and only if \java{ObjectType} ``is-a'' \java{DeclaredType}.  

\Q In~\ref{toyOutput}, 
did the variable's \emph{declared} type or the object's \emph{instantiated} type 
determine the version of \java{makeNoise} that was called? 

\begin{answer}[2em]
The object's instantiated type---notice that 
the variable's declared type is the same in all three instances. 
\end{answer}

\Q Would it ever make sense to construct a \java{LoudToy} object? Why/why not?
\begin{answer}[2em]
Answers will vary, but 
a \java{LoudToy} isn't very useful by itself, 
since all it can do is get and set volume. 
It needs other attributes and methods to represent an actual toy. 
\end{answer}

\clearpage


%%% \model{Abstract Methods}

The \java{abstract} keyword can be used to declare methods that have no body, 
forcing subclasses to override them. 
%These methods must be overridden in subclasses.
Classes with abstract methods must also be defined as abstract.

\begin{quote}
\begin{javalst}
public abstract class LoudToy {
    private int volume;

    public LoudToy(int volume) {
        this.volume = volume;
    }

    public int getVolume() {
        return volume;
    }

    public void setVolume(int volume) {
        this.volume = volume;
        makeNoise();
    }

    public abstract void makeNoise();
}
\end{javalst}
\end{quote}


%%% \quest{15 min}


\Q Summarize the differences between \ref{\currfilename} and your answer to \ref{LoudToyV1}.

\begin{answer}
The class and the \java{makeNoise} method are declared as abstract.
The definition of \java{makeNoise} ends with a semicolon, rather than an empty body \verb|{}|.
\end{answer}


\Q Open \textit{LoudToy.java} (from \ref{\currfilename}) in your IDE.
Remove the word \java{abstract} from the class definition.
What are the two compiler errors?

\begin{answer}
The type LoudToy must be an abstract class to define abstract methods. \\[1ex]
The abstract method makeNoise in type LoudToy can only be defined by an abstract class.
\end{answer}


\Q Replace the word \java{abstract} in the class definition, and then remove the word \java{abstract} from the method definition.
What is the compiler error now?

\begin{answer}
This method requires a body instead of a semicolon.
\end{answer}


\Q Remove the definition of \java{makeNoise} altogether, and notice the compiler error.
Why is it necessary to declare this method in \java{LoudToy}?

\begin{answer}[3em]
The \java{setVolume} method calls the \java{makeNoise} method.
\end{answer}


\Q Undo all changes in \textit{LoudToy.java}, and add the following \java{main} method.
What is the compiler error message?
Why do you think Java doesn't allow you to construct a \java{LoudToy}?

\begin{javalst}
public static void main(String[] args) {
    LoudToy toy1 = new LoudToy(1);
    toy1.makeNoise();
}
\end{javalst}

\begin{answer}
The compiler says, ``Cannot instantiate the type LoudToy.''
Abstract classes cannot be instantiated, because some of their methods aren't implemented.
\end{answer}


\Q Open \textit{ToySheep.java} and rename \java{makeNoise} to \java{makeNoise2}.
What is the compiler error?

\begin{answer}[3em]
The type ToySheep must implement the inherited abstract method LoudToy.makeNoise().
\end{answer}


\Q Rename the method back to \java{makeNoise}, but change \java{void} to \java{int}.
What is the error now?

\begin{answer}[3em]
The return type is incompatible with LoudToy.makeNoise().
\end{answer}


\Q\label{abstractKey2}
Explain how an abstract method is like a contract.

\begin{answer}[5em]
If you inherit an abstract class, you must override the abstract methods exactly as defined.
This is important because they might be called in the code of the abstract class.
\end{answer}


\end{document}
