\model{The \java{static} Keyword}

We have seen the \java{static} Java keyword come up in a few situations. 
It's time for a deep dive into what it means and how to use it. 

{\footnotesize 
\begin{minipage}{0.48\textwidth}
\textbf{Version 1:}
\begin{javalst}
public class StudentV1 {
    private String name;
    private char grade;

    public StudentV1(String name, char grd) {
        this.name = name; 
        this.grade = grd;
    }

    @Override 
    public String toString() {
        return name + " earned " + grade;
    }

    public static void main(String[] args) {
        StudentV1 a = 
            new StudentV1("Adaline", 'A');
        StudentV1 b = 
            new StudentV1("Belicia", 'B');
        StudentV1 c = 
            new StudentV1("Charlie", 'C');
        System.out.println(a);
        System.out.println(b);
        System.out.println(c);
    }
}
\end{javalst}
\end{minipage}\hfill
\begin{minipage}{0.48\textwidth}
\textbf{Version 2:}
\begin{javalst}
public class StudentV2 {
    private String name;
    private static char grade;

    public StudentV2(String name, char grd) {
        this.name = name; 
        StudentV2.grade = grd;
    }

    @Override 
    public String toString() {
        return name + " earned " + grade;
    }

    public static void main(String[] args) {
        StudentV2 a = 
            new StudentV2("Adaline", 'A');
        StudentV2 b = 
            new StudentV2("Belicia", 'B');
        StudentV2 c = 
            new StudentV2("Charlie", 'C');
        System.out.println(a);
        System.out.println(b);
        System.out.println(c);
    }
}
\end{javalst}
\end{minipage}
}

\quest{20 min}

\Q Examine the two versions of a simple \texttt{Student} class above. 
Notice the subtle differences, and predict the output of each. 
\emph{After} predicting, run them (in \texttt{src/student}) to confirm. 
\begin{quote}
\texttt{StudentV1} output: 
\begin{answer}
\begin{javaans}
Adaline earned A
Belicia earned B
Charlie earned C
\end{javaans}
\end{answer}

\texttt{StudentV2} output: 
\begin{answer}
\begin{javaans}
Adaline earned C
Belicia earned C
Charlie earned C
\end{javaans}
\end{answer}
\end{quote}

\Q The English word \emph{static} can take 
\href{https://www.merriam-webster.com/dictionary/static}%
{several different meanings}. 
Based on this example, which meaning do you think Java is using? 
\begin{answer}[3em]
``showing little change'' fits best (or ``stationary''); 
in V2, the static variable \java{grade} is shared 
between all \java{StudentV2} instances. 
Its value is tied to the \emph{class}, 
not individual instances. 
\end{answer}

\Q In the Java API docs for the \java{Integer} class, 
find the list of fields. 
What are some of the \java{static} fields in this class? 
Click on each field's name to see more info. 
\begin{answer}[3em]
\texttt{BYTES}, \texttt{MAX\_VALUE}, \texttt{MIN\_VALUE}, 
\texttt{SIZE}, \texttt{TYPE}. 
Note all are declared as \java{public static final}, 
where the \java{final} keyword means 
their values cannot change once initialized. 
\end{answer}

\Q\label{staticPatterns} Look at \java{static} fields in other Java classes, 
such as \java{Math}, \java{Calendar}, and \java{HttpURLConnection}. 
Based on what you find, 
what do you think are some patterns/conventions/best practices 
for using \java{static} fields? 
\begin{answer}
Notable examples: \java{Math.PI}, 
\java{HttpURLConnection.HTTP_NOT_FOUND}, and 
\java{Calendar.SUNDAY} (and \java{MONDAY}, etc.). 
Patterns: 
(1) Accessed by the syntax \texttt{ClassName.FIELD\_NAME}. 
(2) Usually named with all-caps and underscores. 
(3) Often used to give constants descriptive names. 
\end{answer}

Methods can also be declared as \java{static}, 
meaning they are associated with the class itself and 
not with individual instances. 
In fact, we can call static methods 
(a.k.a.~\href{https://docs.oracle.com/javase/tutorial/java/javaOO/classvars.html}%
{class methods}) without ever creating an instance of the object type. 
A static method has \emph{unchanging} behavior: 
it is not state-dependent, 
meaning it does not do different things depending on 
the data stored in the current object instance. 
Static methods are (almost always) true functions, 
in the mathematical sense: 
for each valid input, they will give the same output every time. 

\Q We have already used one static method in the \java{Integer} class: 
\java{Integer.parseInt(String s)}. 
(Notice the camel-case naming convention and the 
\texttt{ClassName.methodName()} syntax.)
Look at the list of \java{static} methods in \java{Integer}. 
What are some typical use cases for static methods? 
\begin{answer}[2em]
comparing two \java{int} values or finding their max/min, 
counting leading/trailing zeros, 
converting from one numerical representation to another, 
manipulating the binary, etc. 
\end{answer}

\Q We always declare the \java{main} method in Java as \java{static}. 
Why do you think this is necessary? 
\begin{answer}[6em]
If \java{main} was not \java{static}, 
we would have to instantiate an object to call it. 
But we can't run any code (let alone instantiate an object) 
until we start the program by running \java{main}, 
so we would get a 
\href{https://en.wikipedia.org/wiki/Chicken\_or\_the\_egg}%
{chicken-and-egg} problem: 
which came first, the instance or the \java{main} call? 
Java \emph{could} have been designed to start by 
instantiating an object of the main class type and \emph{then} 
calling main, but this can get messy. 
%if our main class has multiple constructors, which should Java call? 
\end{answer}

\Q Examine the provided \emph{Point.java} file. 
Uncomment the print statements, and add the missing methods. 
Which one is static? 
How do the two distance methods differ in ``point'' of view? 
\begin{answer}[5em]
The static method is \texttt{distanceBetween}. 
It represents a ``completely objective third-party observer 
with absolutely no personal interest in the matter'' 
that's looking at \texttt{a} and \texttt{b} 
and measuring the distance in between. 
The non-static (instance) method is \texttt{distanceTo}. 
It represents a ``first-person'' perspective: 
Point \texttt{a} looks at point \texttt{b} and 
measures how far away it is. 
\end{answer}

