\model{Message Chains}

A \emph{message chain} is code in the form 
\begin{center}
\java{someObject.someMethod().otherMethod().stillOtherMethod();}
\end{center}

For example (this is taken from Java AWT/Swing; 
the details of these classes are not important), 
your code might have a line like 
\begin{center}
\java{myFrame.getBufferStrategy().getCapabilities().getFlip().wait(17);}
\end{center}
A rational programmer might split up this message chain 
by introducing intermediate variables:
\begin{javalst}
BufferStrategy strategy = myFrame.getBufferStrategy();
BufferCapabilities capabilities = strategy.getCapabilities();
FlipContents flip = capabilities.getFlipContents();
flip.wait(17);
\end{javalst}

By splitting the message chain like this, 
the dependencies become even clearer, 
as we see in the UML diagram below. 
Your code knows details \emph{four levels deep} 
in the called operations! 

\begin{center}
\includegraphics[width=0.7\linewidth]{message_chain}

(\href{https://www.plantuml.com/plantuml/uml/RKzB3e903Dtt57C1FO8Xah1mwyfbXIADyo5jMH3ZtMKaSQAuQVi-UI-jr51i2Xxqpd54mU3KPa\_B56xVoi0TkK0sf4vNq3FvvORMewgxf4dgjD0FU09yqExWJerW85XN9evJtvESJT3eDiCtB8OQHxooPsDIs5BPs7WUwnIDecdft6ygYlGtlCW\_aHN5BZ\_\_0W00}{PlantUML source})
\end{center}

\quest{30 min}

\Q\label{elimMsgChain} If you could modify these classes however you wanted, 
how could you eliminate this message chain to reduce dependencies? 
Draw your updated design on a whiteboard, paper, or in PlantUML, 
and explain below. 
\begin{answer}[6em]
The solution is usually to embed the required feature in 
the first class in the chain. 
This insulates the caller from the inner classes. 
Then the first class might implement the feature itself, 
or if it still needs to rely on its internals, 
repeat the message chain removal. 
For this example, this process gives 
\href{https://www.plantuml.com/plantuml/uml/ZP2n3e8m48RtFaL77Nm5GoHXS7KmZYSzv69fIU-88SPtjnKYe0jBU\_tkk-PlCnO8qZLMyPrS2m4QpT9RZ6kWyjEfoQ2k2AtXeRxuBkG-5WrAxICjBJ1F-AbU8peMyS84j6QKmb9x1f8DO-cTuMtqHl7GLHM6amGGlFMh1t9euKgMX91N3ALMxfsWaq5\_\_7hVD-psOtWrR0aiGwhbSHBceTG\_Ik8yYChGcVJFRm00}{this design}. 
\end{answer}

\clearpage

\subsection*{Solar System Problem}
\textbf{System description:}  
A Java program draws a minute-by-minute updated diagram 
of the solar system including all planets and moons. 
To update the moon's position, 
the moon's calculations must have 
the updated position of the planet it is orbiting. 
The diagram is colored---all planets are drawn the same color and 
all moons are drawn the same color. 
However, it needs to be possible to modify the planet color or the moon color. 

\Q Examine the UML diagram below of a possible design for this system. 
Which dependencies seem essential? 
Which seem like we might be able to weaken or eliminate them? 
\begin{center}
\includegraphics[width=0.7\linewidth]{solar_system_original}

(\href{https://www.plantuml.com/plantuml/uml/ZL792i8m4BtdAq9FLQnd3oA8NXL1\_82X7JGuMKX6YeZ\_RenRiNMvXCcycRSaOQ-C0mzQ1ZuInjEhaW-QX2W9Gf1hI-3Nny2e5w2CF0afTs0gmfdLxi2un7fbWs9bJSvAOs3GhlUScdkefqJvreFRgJAya8shW755O91dbgpF3TQfU9zPM6lQ4-SEYz6UY53vmrhLCdM-Sztif9JkvQnnIyHpOFXJfaFZ6YSfw9IhoQWVIMCjklXRDWRokwHelavHJ6GsokufHJAIzJCvBXwjz--E2urM2nvEFYr6v4f\_0G00}{PlantUML source})
\end{center}
\begin{answer}[3em]
The SolarSystemMain dependencies on Moon and Planet seem unnecessary, 
but the rest are essential. 
\end{answer}

\Q Investigate the implementation of this design 
provided in the \texttt{src/solarSystem} package. 
For the seemingly unnecessary dependencies, 
where do they appear in the code? 
What principle(s) do these dependencies violate? 
\begin{answer}
SolarSystemMain depends on Moon and Planet via 
the message chains in the \java{handleUpdatePlanetColor} 
and \java{handleUpdateMoonColor} methods. 
The use of non-enhanced for loops is intentional 
to highlight the message chains. 
\end{answer}

\Q Improve the UML design by applying our standard procedure 
for handling the type of dependency you found. 
Use the PlantUML source linked above as a starting point, and 
paste your updated PlantUML link below. 
(Remember to click the ``Decode URL'' button after making changes.)
\begin{answer}
Improved design, derived from mechanically following the advice 
to propagate the method into the first class of the chain: 
\href{https://www.plantuml.com/plantuml/uml/bP9DQyCm38Rl-HKcftGiznv6ORJNbaBP3n2EQCta3ooLK4R\_UpqtQTAcxEDYZFHQxoF9\_6bSW0XMiPm8qncDbgEbej04p6hd2UBkn89s3SQfhqAf0xY6SEbjE0fkiDrwDcAygyHOP04RPZVLAaBxJbJj8uI3qJnaFa83Wbi2XxBqv6FbiCsNHytQUpTLd9yWPZn66LVL\_OCK7ohTelgNDff90-VEzsgIBKFdbqkjG20NJWv\_DBWR3p-RnntyaEXPldEPn6RHcED7aWdug\_etN5h4U0LrlL3blEp6KsIoygLKRtKVUlnJwS4eT040}{PlantUML link}
\end{answer}

\Q Revisit the system description. 
What do you notice about the color specifications 
that suggests a further design improvement? 
\begin{answer}[5em]
The system description has all planets sharing the same color 
and all moons sharing the same color, 
so we only need one \java{planetColor} and one \java{moonColor} 
for the SolarSystem, which can be stored as fields. 
The current design duplicates data by storing 
the same moon color in all Moon objects and 
the same planet color in all Planet objects. 
\end{answer}

\Q Refactor the solar system project to match 
\href{https://www.plantuml.com/plantuml/uml/ZPBB2i8m44Nt-OhGLPLsxq84qQqY53zWQ0SDJXua8wM8\_swDZrOhwYPCvd7lcP1mKCQ11oq3DuJjwLN9Hqr2b0GXoC8I-A89Z7e5oiYJa78FfY9SMEsEZ6kiDpOeLjQah3G61kr6pwwbXtfEbEuykBqgGrVPkWeODmG6UM79-jHW7OFtdfMrPjXn\_e0OyLmdsTxqOxYDon\_8rG3sVFUuORx8HwabCFmf\_5JD-eHP1zYvNHUENhVQ3wS1K2Q\_cR\_uYVxZvVbC9fFalgo85CauaTFyaNh\_3Ur0BtK1}{this design}, 
which includes the improvement from the previous question. 
You may get help from GenAI tools. 
Looking at the methods where you previously saw 
evidence of unnecessary dependencies, 
what does the new implementation do? 
How does it avoid the bad dependencies of the original version? 
\begin{answer}
Each of SolarSystemMain's \java{handleUpdatePlanetColor} and 
\java{handleUpdateMoonColor} methods should now be a one-liner 
that calls the appropriate setter on \java{this.solarSystem}. 
This prevents SolarSystemMain from needing to know that 
the Planet and Moon classes exist. 
\end{answer}