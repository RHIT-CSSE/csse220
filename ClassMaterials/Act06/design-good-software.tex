\model{The Software Design Process}

\quest{30 min}

\Q Summarize from the classwide discussion: 
What makes software good? 
\begin{answer}[2.2em]
Answers will vary slightly based on student input, 
but should be along the lines of: 
works correctly (functional), 
easy to extend/modify/reuse, 
readable, 
simple (as possible), 
efficient
\end{answer}

\textbf{Interlude: Requirements and Design}

The software development process 
typically begins with abstract ideas, 
such as ``dog-walking gig economy marketplace''. 
These ideas need to be formed into \emph{requirements}---%
what the clients want from the software. 
(CSSE371 will teach you this \emph{requirements analysis} step.) 
Once we have clear requirements, 
we need to design software to meet them. 
In CSSE220, we start practicing the design process. 
You will build on these skills in future courses, 
including CSSE280 (designing web applications), 
CSSE333 (designing databases and applications which interact with them), 
CSSE374 (applying more advanced software design patterns), 
and senior capstone. 

\Q %For all practical software systems, 
If we come up with multiple design options, 
how can we figure out which one to use? 
Brainstorm some ideas. 
\begin{answer}[4em]
We could implement several possible designs, 
then compare them at the end, but that would waste a lot of time. 
Instead, we could ``sketch out'', in some way, 
several candidate designs, and try to see which is best 
\emph{without} implementing them all. 
\end{answer}

\Q Software design is an \emph{iterative} process 
in which we \emph{learn from failure}, 
refining our design until we land on something workable. 
To make the process efficient, 
we need some concise way of describing candidate designs. 
In the specific context of object-oriented design, 
how could we accomplish this? 
(Hint: we have seen a useful type of diagram in previous classes.)
\begin{answer}[2.5em]
We can use UML diagrams to summarize designs and 
visualize several classes together. 
We will add arrows between classes in the UML diagram 
to represent their relationships. 
\end{answer}

\Q Sketch (on a whiteboard or paper) 
a UML diagram for the two classes below. 
Refer to past worksheets as needed. 
Make sure to capture the relationship between the two classes. 

{\small
\begin{minipage}{0.48\textwidth}
\begin{javalst}
public class Ninja {
    private String name;
    private int level;

    public Ninja(String name) {
        this.name = name;
        this.level = 1;
    }

    public void setLevel(int level) {
        this.level = level;
    }
}
\end{javalst}
\end{minipage}
\hfill
\begin{minipage}{0.48\textwidth}
\begin{javalst}
public class Pirate {
    private Ninja enemy;

    public Pirate(Ninja enemy) {
        this.enemy = enemy;
    }

    public Pirate() {
        this.enemy = new Ninja("Foo");
    }
}
\end{javalst}
\end{minipage}
}

\Q\label{key1} Compare your UML diagram against 
\href{https://www.plantuml.com/plantuml/uml/HOrB2W8n343tEKMMYlG2NSWSG4Jm164cI3IDqiG16RoxhH\_SXEGzU1aygCQzAFX5x4QL2deybDkiChuDZ4hkk1Sx4owGZ0fdF3Pl9qZACsj6iO2DS-pwkVh3TSRvAXCyVvc3L0h6fPFldhQVEC30DhMNBm00}{this solution}. 
What differences do you notice? 
\begin{answer}[5em]
Answers will vary. 
You may have included the constructors as methods, 
but we often omit these. 
The biggest difference is probably that 
you included \java{enemy} as a field in \java{Pirate} 
instead of drawing an arrow from \java{Pirate} to \java{Ninja}. 
We read this arrow as ``Pirate \emph{has-a} Ninja''. 
This arrow implies the existence of 
the field of type \java{Ninja} in \java{Pirate}. 
\end{answer}

\Q Add the below \java{Wizard} class to your UML diagram. 
{\small
\begin{javalst}
public class Wizard {
    private String name;
    private ArrayList<Ninja> ninjaFriends;

    public Wizard(String name) {
        this.name = name;
        this.ninjaFriends = new ArrayList<>();
    }

    public void addNinjaFriend(Ninja newFriend) {
        this.ninjaFriends.add(newFriend);
    }
}
\end{javalst}
}
\begin{answer}[3em]
See \href{https://www.plantuml.com/plantuml/uml/PO-\_2i9038VtF4Ner5hk1MuedPn410Vdq0iIlKQvEojQV7VxfzB19IJV3xuaTKmKqgFr4A-YTmhKOamlpqCDqam9D9vYnBteXR07ezIonUEGwncCvuwzHT44PUIq6yTyXON5xYOEFZ\_DGG8bndub9tbJS7\_cafoRTcw3iBfS-JbtTZwb60sBpLGpWyLbgcoJBQXcTSCRNm00}{solution}. 
The ``*'' on the arrow means \java{Wizard} stores 
\emph{0 to many} references to \java{Ninja} objects. 
\end{answer}

\Q We will use 
the \href{https://www.plantuml.com/plantuml/}{PlantUML} web app 
to create UML class diagrams throughout the course. 
\textbf{In pairs} (or solo if you only have three today), 
practice using this tool by reproducing the below UML class diagram. 
Use the above example and your UML cheatsheet for reference. 
(Note: We will always add \texttt{skinparam style strictuml} 
as a first line to keep our UML diagrams simple.)
\begin{center}
\includegraphics[width=0.45\textwidth]{main_plantuml}
\end{center}
\begin{answer}[2em]
\href{https://www.plantuml.com/plantuml/uml/NOx12i8m38RlVOgmqnHsyYxhxjx4UzYA5hjCcbGH4Nm7tz0diOrCn4lwatuV-LiMZ9B6OFZewOGHHs2v1fTdzBt-c3uWCspG4zmCWAPwKH8xiI5qEmp9yNPQvUrzvgsoXAChhwrBcgMFKP\_B\_54wLVefaPJGVO\_yunQWYNv\_a6QTCQFLNe\_dC5qeRwfb1TQOrj6W\_Ty0}{PlantUML solution}
% @startuml
% skinparam style strictuml

% class Main {
%    Main()
%    setAllBValuesTo3()
% }

% class A{
%    name
%    A(name)
%    setBValue(value)
% }

% class B{
%    count
%    B()
%    setValue(value)
% }

% A  -right->  B
% Main –down-> "*" A
% @enduml
\end{answer}