\model{Object-Oriented Design Principles 1 and 2}

Object-oriented design principles (OODPs) are guidelines 
(think, ``rules of thumb'') that help developers write good software. 
Today, we will get familiar with the first two OODPs from our CSSE220 list 
(see the OODP reference sheet in Moodle). 

\quest{25 min}

\Q The most important object-oriented design principle (OODP \#1) is\dots
\begin{answer}[4em]
functionality! 
Our design must 
(a) be able to \emph{store required information} 
(one/many to one/many relationships), 
(b) be able to \emph{access the required information} 
to accomplish tasks, and 
(c) \emph{not duplicate data} (IDs/object references are OK) 
\end{answer}

\Q A second priority (OODP \#2) is 
to structure the design \emph{around the data} to be stored. 
Based on the examples of classes we have created so far, 
which \href{https://en.wikipedia.org/wiki/Part\_of\_speech}{part of speech} 
usually signals we should consider creating a class? 
\begin{answer}[3em]
Nouns, because each class is a blueprint for some type of object. 
Note that we will generally not use plural nouns as class names; 
instead, we will create multiple instances of an object when needed. 
\end{answer}

\Q Part (b) of OODP \#2 states, 
``Classes should have intelligent behaviors (methods) 
that may operate on their data.''
Why do you think this is part of OODP \#2? 
% How does this relate to real-world objects? 
\begin{answer}[4em]
It makes sense to put methods that use certain data 
in the same place as those data for easy access. 
And most interesting real-world objects 
have not only internal ``data'' but also behaviors using that data. 
(For example, a coffee maker is more than just a container for grounds.)
\end{answer}

\textbf{Interlude: Design Problems}

Throughout the course, we will practice object-oriented design 
using story problems: (multi-)paragraph descriptions of system requirements. 
We will practice the design process by 
\begin{enumerate}
\item identifying flaws in designs we give you, and 
\item developing your own design(s), following the OODPs. 
\end{enumerate}
Since object-oriented design is such an important skill, 
there will be several opportunities to practice in class, 
extra practice problems that you can work through on your own time, 
and assignment/exam design problems, including your final project design. 

\Q \textbf{Design Problem 1: Book Tracker}
\begin{quote}
A website tracks books and the kids that read them. 
For each book the system stores the name and author. 
For each kid the system stores name and grade level. 
The teacher enters when a kid reads a particular book. 
It should be possible to print a report on a book 
that includes all kids who have read a particular book 
(with their grade level). 
It should be possible to print a report on a kid 
that includes the books (with authors) a particular kid has read. 
\end{quote}
Suppose we start with a class called \java{BookMain} 
that takes care of user input via three methods: 
\java{handleNewReading(String bookName, String kidName)}, 
\java{handlePrintReportForKid(String kidName)}, and 
\java{handlePrintReportForBook(String bookName)}. 
\begin{enumerate}
\item What classes should we add? And with what fields? 
\begin{answer}[4em]
Looking at the nouns in the description, 
``book'' and ``kid'' both make sense to turn into classes. 
Our \java{Book} class should have \java{name} and \java{author} fields. 
Our \java{Kid} class should have \java{name} and \java{gradeLevel} fields. 
\end{answer}
\item What relationships do we need to represent between classes? 
\begin{answer}[4em]
\java{BookMain} should probably keep a list of all \java{Kid} objects 
and a list of all \java{Book} objects. 
Each \java{Kid} might need to keep a list of its read \java{Book} objects, and 
each \java{Book} might need to keep a list of the \java{Kid} objects who have read it. 
\end{answer}

\item What methods should our new classes have? 
\begin{answer}[3em]
Both \java{Kid} and \java{Book} need a \java{printReport()} method. 
Also, \java{Kid} needs an \java{addBook(Book book)} method, and 
\java{Book} needs an \java{addKid(Kid kid)} method. 
\end{answer}

\item Create a UML class diagram for your design using 
\href{https://www.plantuml.com/plantuml}{PlantUML}. 
(Remember to add the \texttt{skinparam style strictuml} line at the top.)
To save your design, click the ``Decode URL'' button, 
then copy/paste the updated URL from your browser's address bar. 
\begin{answer}[2em]
\href{https://www.plantuml.com/plantuml/uml/VO\_FQeGm48VlynJ1avRs3KgH7dffAyKtc3g31lD79h6bb7ttJKHTsOMzXCnlviitgKD4YRCr44Rj9XIqAiG\_m-aKtIqTw0o6e5wz7pzHE\_KF0peotF1loqZQzULtwZLe-L6DclAbMcU-HBlOykGbldbvrrJU7SYE\_R4AJfixWKdhKWYz8F47\_x21wSAM5I3HZX\_Gd6qipd7mSikjUotK\_isdb-AXoERhA9UGDqh5yo62cXtDrfm1}{PlantUML solution}
\end{answer}

\item\label{key2} Consider \href{https://www.plantuml.com/plantuml/uml/VOzDQiCm48NtSmhXbLtQ6vHYkkYchIc-mTGpsCBwCIEv9OJSFP9ZEu50Dc9uypvzKXqYIfojWJ1gDw6WLI4U3ATJTBTie3COWdhtVln6xTGH1dHakE7\_bf6qwylVj6lGyhCQDUMXMZi\_ebriUV8IFxnyQYeV5h9Zlwo2q-PEODAw548lIFp5VsnWkh9b1KYqupTepZOMvpXukUVMTotK\_isNj-Afo6R8zZI\_Nf9SW9eTpTQS0G00}{Bad Design A}. 
Applying OODPs 1 and 2, what is wrong with this design? 
\begin{answer}[3em]
This design does not function. 
There is no (sane) way to look up a book 
for printing a report or for associating with a Kid. 
\end{answer}

\item\label{key3} Consider \href{https://www.plantuml.com/plantuml/uml/VP3FJe0m3CRlVOg54nXz1OCUzA0ImnjKsi32s4WtDCRuxcu3fj74oz8\_xU\_xrjO7bB0j1loixOg2Y\_BXtN1yHG-v0uD1xzMjS\_CJQgi-O49BXZj-wnb9sx5-YRqE5xvKiwOKDCVCisWRUbwTX3id3vhgVmUIHe4ry7bgnyKeQsCHHa7YHtvb0-UWJpPoThE5oScUhz\_akW4aAl0Vu5GxoPYN8HhS0IcuBsh-B\_3uJiQLwrnTLnTLCloxb59esLAywHS0}{Bad Design B}. 
Applying OODPs 1 and 2, what is wrong with this design? 
\begin{answer}[4em]
This design functions but there is a lot of data duplication, 
which in general we want to avoid. 
In particular, the author/title information in Kid is duplicated, 
and the name/grade level information in the book is duplicated. 
\end{answer}

\item Revisit your original design. 
If it had similar flaws to those in Bad Designs A and B, fix them. 
If it didn't have those design flaws, 
double-check that your design adheres to OODPs 1 and 2. 
Provide a link to your final PlantUML design below. 
\begin{answer}[2em]
\href{https://www.plantuml.com/plantuml/uml/VO\_FQeGm48VlynJ1avRs3KgH7dffAyKtc3g31lD79h6bb7ttJKHTsOMzXCnlviitgKD4YRCr44Rj9XIqAiG\_m-aKtIqTw0o6e5wz7pzHE\_KF0peotF1loqZQzULtwZLe-L6DclAbMcU-HBlOykGbldbvrrJU7SYE\_R4AJfixWKdhKWYz8F47\_x21wSAM5I3HZX\_Gd6qipd7mSikjUotK\_isdb-AXoERhA9UGDqh5yo62cXtDrfm1}{PlantUML solution}
\end{answer}

\end{enumerate}

\emph{Aside: If you want a detailed, step-by-step process 
for parsing story problems and identifying 
classes, fields, and methods, 
see the BookTracker slides in today's class materials folder.}

