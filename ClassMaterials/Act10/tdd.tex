\model{Test-Driven Development (TDD)}

While working on a fun side project, 
we often want to jump into writing code right away 
(spending as little time on the design phase as possible) 
and maybe later think about testing our code. 
For more serious work, 
such as when a hospital or aircraft needs our code 
to function correctly all the time, 
we should consider the technique of 
\emph{Test-Driven Development} (TDD). 
The TDD process is:
\begin{enumerate}
\item Write a new unit test (that fails) for a not-yet-implemented system requirement. 
\item Write just enough code to pass the new test. 
\item \href{https://en.wikipedia.org/wiki/Code_refactoring}{Refactor} 
    to better follow object-oriented design principles. 
\item Repeat for other system requirements. 
\end{enumerate}

\begin{figure}[!ht]
\centering
\includegraphics[width=0.7\textwidth]{tdd-fowler}%
\caption{A flowchart of Test-Driven Development \href{https://martinfowler.com/bliki/TestDrivenDevelopment.html}{by Martin Fowler}.}%
\label{fig:tdd}
\end{figure}

Continuing with our \java{BadFrac} class, 
we will practice TDD. 

\quest{20 min}

\Q Suppose we are asked to add a new feature to our \java{BadFrac} class: 
multiplication. 
In particular, our new method header should be 
\java{public BadFrac multiply(BadFrac incoming)}. 
Add this method to \java{BadFrac.java}, 
with only one line: 

\java{throw new UnsupportedOperationException("Not yet implemented");}

\Q Using the same IDE test generation process as in~\ref{create-tests.tex}, 
create a \java{testMultiply()} method in \emph{BadFracTest.java}. 

\Q Add some \java{assert} statements in \java{testMultiply()}, 
following our guidelines from~\ref{create-tests.tex}. 

\Q Run your new \java{testMultiply()} test case. 
Do you get the results you expect? 
\begin{answer}[2em]
The first \java{assert} statement should fail because of 
the \java{UnsupportedOperationException}. 
\end{answer}

\Q Splitting into pairs (or solo if three), 
implement the \java{multiply} method, 
adjusting it until it passes your unit tests. 
Then, compare your solution with your teammates' solution(s). 

\Q\label{critiqueTDD} 
What do you think are some advantages and disadvantages of TDD? Explain. 
\begin{answer}
See \href{https://en.wikipedia.org/wiki/Test-driven_development#Advantages_and_Disadvantages}{Wikipedia}
for a nice list. 
Some pros: clarifies requirements, increases confidence and coverage, 
many find it boosts productivity. 
Some cons: more code volume (lots of lines of tests), 
more maintenance required, learning curve, may incentivize overcomplicated code. 
\end{answer}



