\model{Unit Testing}

In previous assignments, we have provided you with unit tests 
(using the JUnit Java framework) to help you verify that your code is working as expected. 
Today, we look at how (and why) to create this unit tests from scratch. 

Take a look at the files \emph{BasicRectangle.java} and \emph{BasicRectangleTest.java} 
in \texttt{src/rectangle} and answer the following questions. 

\quest{15 min}

\Q Examine the methods in \emph{BasicRectangleTest.java}. 
What details are present that you do not see in typical source code methods, 
such as those in \emph{BasicRectangle.java}? 
What do you \emph{not} see here that you expect to find in source code classes? 
\begin{answer}
The methods have an annotation above them, \java{@Test}. 
They also call \java{assertEquals()} methods 
which are imported from \java{org.junit.Assert}. 
There are no explicit fields or constructors. 
\end{answer}

\Q Look up a dictionary definition of ``assert''. 
How does it apply to the \java{assertEquals()} methods 
you see in JUnit tests? 
Why do you think the JUnit creators chose ``assert'' 
over other words? 
\begin{answer}[4em]
The ``assert'' definition from Merriam-Webster is 
``to state or declare positively 
and often forcefully or aggressively''. 
Since the test case fails immediately when its first assert fails, 
throwing an \java{AssertionError}, 
it makes sense to use a word with such a forceful connotation 
(instead of ``check'', ``test'', or similar). 
\end{answer}

\Q Run \emph{BasicRectangleTest.java} in your IDE as JUnit tests. 
Which tests fail, and which tests pass? 
\begin{answer}[3em]
\java{testCalculatePerimeter()} and \java{testToString()} pass; 
the others, \java{testCalculateArea()} and 
\java{testCalculateDiagonalLength()}, fail. 
\end{answer}

\Q Search the \href{https://junit.org/junit4/javadoc/latest/}{JUnit 4 API} 
for the definitions of the \java{assertEquals} methods. 
(Hint: Look at the \java{import} statement at the top of \emph{BasicRectangleTest.java}.)
Which versions of \java{assertEquals} are we using here to test \java{BasicRectangle}? 
\begin{answer}[4em]
We are using \java{assertEquals(long expected, long actual)} to compare integer values, 
\java{assertEquals(Object expected, Object actual)} to compare String values, and 
\java{assertEquals(double expected, double actual, double delta)} to compare double values. 
\end{answer}

\Q In the \java{assertEquals} calls within \java{testCalculateDiagonalLength()}, 
what does the third parameter do? 
Experiment with setting it to different values. 
Can you adjust this parameter in a way that makes \java{testCalculateDiagonalLength()} fail earlier? 
\begin{answer}[2em]
This ``delta'' parameter sets the tolerance for the floating-point comparison. 
Changing the first delta from \texttt{0.0001} to \texttt{0.00001} causes the assertion to fail. 
\end{answer}

\Q\label{whyUnitTests}%
Instead of having all of this extra syntax and overhead for JUnit tests, 
we could just print expected and actual values to the console and compare them. 
Why do you think using unit test frameworks is a best practice? 
\begin{answer}[3em]
IDEs can format JUnit results in an easy-to-read, easy-to-navigate manner. 
Printouts would require analyzing all lines and spotting subtle differences, 
such as a single-character difference between long strings. 
\end{answer}

